
\section{A Profile for Safety Critical Java}
\label{sec:scjava}

The proposed profile is an refinement of the profile described in
Section~\ref{sec:rtprof}. Further development of applications on JOP
shall be based on this profile and the current applications (e.g.\
\"OBB bg and Lift) should be migrated to this profile.

\subsection{Introduction}

Safety-critical Java (SCJ) builds upon a broad research base on using
Java (and Ada) for hard real-time systems, sometimes also called high
integrity systems. The Ravenscar profile defines a subset of Ada to
support development of safety-critical systems~\cite{289525}. Based
on the concepts of Ravenscar Ada a restriction of the RTSJ was first
proposed in \cite{Pusch01}. These restrictions are similar to SCJ
level 1 without the mission concept. The idea was further extended in
\cite{ravenscar:java} and named the Ravenscar Java profile (RJ). RJ
is an extended subset of RTSJ that removes features considered unsafe
for high integrity systems. Another profile for safety-critical
systems was proposed within the EU project HIJA \cite{hija}.

PERC Pico from Aonix~\cite{perc:pico:um} defines a Java environment
for hard real-time systems. PERC Pico defines its own class hierarchy
for real-time Java classes which are based on the RTSJ libraries, but
are not a strict subset thereof. PERC Pico introduces stack-allocated
scopes, an elaborate annotation system, and an integrated static
analysis system to verify scope safety and analyze memory and CPU
time resource requirements for hard real-time software components.
Some of the annotations used to describe the libraries of the SCJ are
derived indirectly from the annotation system used in PERC Pico.

Another definition of a profile for safety-critical Java was
published in \cite{jop:scjava}. In contrast to RJ the authors of that
profile argue for new classes instead of reusing RTSJ based classes
to avoid inheriting unsafe RTSJ features and to simplify the class
hierarchy. A proposal for mission modes within the former profile
\cite{jop:scjmodes} permits recycling CPU time budgets for different
phases of the application. Compared to the mission concept of SCJ
that proposal allows periodic threads to vote against the shutdown of
a mission. The concept of mission memory is not part of that
proposal.


In this chapter we focus on the safety-critical Java (SCJ)
specification, a new standard for safety critical applications
developed by the JSR 302 expert group \cite{scj:as:proceedings}. We
should note that the JSR 302 has not been finalized, thus we present
the version from \cite{jop:scjmodes}. When the JSR 302 stabilizes it
will be implemented for JOP and will be the future version of the
real-time profile.

% Text from the SCJ paper
%This draft JSR 302 standard, like previous work, defines a strict
%subset of the RTSJ which is intended to provide a programming model
%suited to a large class of safety critical applications. Restricting
%the features of the RTSJ is intended to make programs more amenable
%to worst case analysis and manual or automatic validation. The SCJ
%is structured in three increasingly expressive levels: Level 0
%restricts applications to a single threaded cyclic executive, level
%1 assumes a single ``mission'' with a static thread assignment, and
%level 2 is a multi-mission model with dynamic thread creation. This
%paper focuses on level 1 which is expected to cover a large number
%of existing SC applications. It should be noted that while all
%levels are designed to run on a vanilla RTSJ VM, it is expected that
%vendors will provided implementations that are optimized for the
%particular features of each level.

\begin{lstlisting}[float=t!,caption={Periodic thread definition for SCJ},
label=lst:scjdef]

package javax.safetycritical;

public abstract class RealtimeThread {

    protected RealtimeThread(RelativeTime period,
        RelativeTime deadline,
        RelativeTime offset, int memSize)

    protected RealtimeThread(String event,
        RelativeTime minInterval,
        RelativeTime deadline, int memSize)

    abstract protected boolean run();

    protected boolean cleanup() {
        return true;
    }
}

public abstract class PeriodicThread
        extends RealtimeThread {

    public PeriodicThread(RelativeTime period,
        RelativeTime deadline,
        RelativeTime offset, int memSize)

    public PeriodicThread(RelativeTime period)
}
\end{lstlisting}

\subsection{SCJ Level 1}

Level 1 of the SCJ requires that all threads be defined during an
initial \emph{initialization} phase. This phase is run only once at
virtual machine startup. The second phase, called the \emph{mission}
phase, begins only when all threads have been started. This phase
runs until virtual machine shutdown. Level 1 supports only two kinds
of schedulable objects: periodic threads and sporadic events. The
latter can be generated by either hardware or software. This
restrictions keeps the schedulability analysis simple. In SCJ
priority ceiling emulation is the default monitor control policy.
The default ceiling is top priority.

The Java \code{wait} and \code{notify} primitives are not allowed in
SCJ level 0 and 1. This further simplifies analysis. The consequence
is that a thread context switch can only occur if a higher priority
thread is released or if the current running thread yields (in the
case of SCJ by returning from the \code{run()} method).

In the RTSJ, periodic tasks are expressed by unbounded loops with,
at some point, a call to the \code{waitForNextPeriod()} (or
\code{wFNP()} for short) method of class \code{RealtimeThread}. This
has the effect of yielding control to the scheduler which will only
wake the thread when its next period starts or shortly thereafter.
In SCJ, as a simplification, periodic logic is encapsulated in a
\code{run()} method which is invoked at the start of every period of
a given schedulable object. When the thread returns from
\code{run()} it is blocked until the next period.

Listing~\ref{lst:scjdef} shows part of the definition of the SCJ
thread classes from \cite{jop:scjava}. Listing~\ref{lst:per} shows
the code for a periodic thread. This class has only one \code{run()}
method which performs a periodic computation.

\begin{lstlisting}[float=t,caption={A periodic application thread in SCJ},
label=lst:per]

new PeriodicThread(
    new RelativeTime(...)) {

        protected boolean run() {
            doPeriodicWork();
            return true;
        }
};
\end{lstlisting}

The loop construct with \code{wFNP()} is not used. The main
intention to avoid the loop construct, with the possibility to split
application logic into \emph{mini} phases, is simplification of the
WCET analysis. Only a single method has to be analyzed per thread
instead of all possible control flow path between \code{wFNP()}
invocations.

We contrast the SCJ threading with Listing~\ref{lst:rtsj:per} where a
periodic RTSJ thread is shown. Suspension of the thread to wait for
the next period is performed by an explicit invocation of
\code{wFNP()}. The coding style in this example makes analysis of the
code more difficult than necessary. First the initialization logic is
mixed with the code of the mission phase, this means that a static
analysis may be required to discover the boundary between the startup
code and the periodic behavior. The code also performs mode switches
with calls to \code{wFNP()} embedded in the logic. This makes the
worst case analysis more complex as calls to \code{wFNP()} may occur
anywhere and require deep understanding of feasible control flow
paths.  Another issue, which does not affect correctness, is the fact
that object references can be preserved in local variables across
calls to \code{wFNP()}. As we will see later this has implications
for the GC.

\begin{lstlisting}[float=t,caption={Possible logic for a periodic thread in the RTSJ},
label=lst:rtsj:per]

    public void run() {

        State local = new State();
        doSomeInit();
        local.setA();
        waitForNextPeriod();

        for (;;) {
            while (!switchToB()) {
                doModeAwork();
                waitForNextPeriod();
            }
            local.setB();
            while (!switchToA()) {
                doModeBWork();
                waitForNextPeriod();
            }
            local.setA();
        }
    }
\end{lstlisting}

The notation of a \code{run()} method that is invoked periodically
also simplifies garbage collection (see Chapter~\ref{chap:rtgc}).
When the GC thread is scheduled at a lower priority than the
application threads it will never interrupt the \code{run()} method.
The GC will only execute when all threads have returned from the
\code{run()} method and the stack is empty. Therefore, stack scanning
for roots can be omitted.
